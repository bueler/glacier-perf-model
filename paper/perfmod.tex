\documentclass[twocolumn,letterpaper]{igs}
%\documentclass[twocolumn]{igs}  % A4 paper needs no option
%\documentclass[review,letterpaper]{igs}

\usepackage{verbatim,xspace,amsmath,amssymb,bm,multirow,dashbox}
\usepackage{tikz}
\usetikzlibrary{arrows}
\usepackage{igsnatbib,lineno}

\usepackage[kw]{pseudo}
\pseudoset{left-margin=0mm,topsep=5mm,idfont=\texttt}

\usepackage{hyperref}
\hypersetup{pdfauthor={Ed Bueler},
            pdfcreator={pdflatex},
            colorlinks=true,
            citecolor=purple,
            linkcolor=red,
            urlcolor=blue,
            }

% math macros
\newcommand\bb{\mathbf{b}}
\newcommand\bc{\mathbf{c}}
\newcommand\bbf{\mathbf{f}}
\newcommand\bg{\mathbf{g}}
\newcommand\bn{\mathbf{n}}
\newcommand\bq{\mathbf{q}}
\newcommand\bu{\mathbf{u}}
\newcommand\bv{\mathbf{v}}
\newcommand\bx{\mathbf{x}}
\newcommand\by{\mathbf{y}}

\newcommand\bA{\mathbf{A}}
\newcommand\bF{\mathbf{F}}
\newcommand\bH{\mathbf{H}}
\newcommand\bN{\mathbf{N}}
\newcommand\bQ{\mathbf{Q}}
\newcommand\bU{\mathbf{U}}
\newcommand\bV{\mathbf{V}}
\newcommand\bW{\mathbf{W}}
\newcommand\bX{\mathbf{X}}

\newcommand\bzero{\bm{0}}

\newcommand{\Div}{\nabla\cdot}
\newcommand\eps{\epsilon}
\newcommand{\grad}{\nabla}
\newcommand{\ip}[2]{\ensuremath{\left<#1,#2\right>}}
\newcommand\lam{\lambda}
\newcommand\lap{\triangle}
\newcommand\RR{\mathbb{R}}
\newcommand{\half}{\tfrac{1}{2}}

\newcommand{\rhoi}{\rho_{\text{i}}}
\newcommand{\pp}{{\text{p}}}
\newcommand{\qq}{{\text{q}}}
\newcommand{\rr}{{\text{r}}}

\newcommand{\mR}{R^{\bm{\oplus}}}
\newcommand{\iR}{R^{\bullet}}

\newcommand{\sold}{s_{\text{o}}}
\newcommand{\told}{t_{\text{o}}}

\newcommand{\byold}{\by_{\text{o}}}

\newcommand{\onecol}[1]{\includegraphics[width=86mm]{#1}}
\newcommand{\onecolless}[1]{\includegraphics[width=80mm]{#1}}
\newcommand{\onecolthin}[1]{\includegraphics[width=75mm]{#1}}

%\newcommand{\twocol}[1]{\includegraphics[width=178mm]{#1}}
\newcommand{\twocol}[1]{\includegraphics[width=165mm]{#1}}
\newcommand{\twocolless}[1]{\includegraphics[width=140mm]{#1}}

\newcommand{\ds}{\displaystyle}

\begin{document}

\title[Performance models for high-resolution ice sheet simulations]{Letter: Performance models for \\ high-resolution ice sheet simulations}

\abstract{Modern numerical ice sheet models, with horizontal resolutions of a few kilometers or less, compute evolving ice geometry and velocity fields using various widely-discussed stress-balance approximations and boundary conditions.  The practical performance of such numerical models is, however, often dominated by the stability restrictions of currently-standard explicit time-stepping.  Such models take many short, stability-limited time steps because ice thickness is updated following each velocity solution.  We clarify the performance of numerical model choices, in the high-resolution limit, by quantifying computational effort in terms of major parameters.  The potential performance of coming generations of Stokes dynamics and/or implicit time-stepping numerical models is assessed.}

\author{Ed Bueler}

\affiliation{Dept.~Mathematics and Statistics, University of Alaska Fairbanks, USA \\
E-mail: \emph{\texttt{elbueler\@@alaska.edu}}}

%\keywords{}

\maketitle

\sectionsize

\subsection{Introduction}

Numerical ice sheet models with evolving ice geometry are now in common use for scientific questions such as quantification of future sea level rise from changes in the Antarctic \citep{Seroussietal2020} and Greenland \citep{Goelzeretal2020} ice sheets, interpretation of the paleoglacial record \citep{Weberetal2021}, and quantification of long-term glacial erosion \citep{SeguinotDelaney2021}.  It is generally accepted that horizontal mesh or grid cells must be smaller than about 10 km in order to generate valid results.  Whether using local mesh refinement \citep[for example]{Hoffmanetal2018} or not, resolutions of a few kilometers to one kilometer \citep{SeguinotDelaney2021} or less \citep{Aschwandenetal2019} have been used at ice sheet scale.  With available computational resources, sub-kilometer meshes will become increasingly common even for long-term, whole ice sheet scale studies.

The current generation of ice sheet models use a variety of stress balances, from simple approximations like the shallow ice approximation (SIA), through ``hybrid'' \citep[for example]{Winkelmannetal2011} and higher-order approximations \citep{GreveBlatter2009}, to the non-shallow Stokes approximation.  With very few exceptions, however, current models alternate between solving the stress balance for velocity, using geometry from the previous time step, and then updating the ice thickness and/or surface elevation.  The latter geometry-update operation, using the mass continuity or surface kinematical equations \cite{GreveBlatter2009}, is where interaction between the ice sheet and the climate is (primarily) simulated, especially through surface mass balance (SMB) and calving processes.  Such schemes are called \emph{explicit time-stepping} for the coupled mass and momentum system which describes ice dynamics.

For simpler partial differential equation problems, the conditional stability of explicit time-stepping schemes is well-understood \citep{LeVeque2007}.  The stability limitations of SIA numerical models have been understood, at least empirically, for some time \citep{HindmarshPayne1996}.  More recent studies have focussed on time-step limits applicable to models using hybrid, higher-order, and Stokes dynamics \citep{Chengetal2017,Robinsonetal2022} or on techniques to mitigate such limits \citep{LofgrenAhlkronaHelanow2021}.

FIXME POSSIBILITY OF IMPLICIT SOLVERS \citep{Bueler2016,WirbelJarosch2020}

FIXME see \citep{GreveBlatter2009,SchoofHewitt2013} for SIA and Stokes equations; CONFUSION FROM TWO FORMATS
    $$\frac{\partial H}{\partial t} + \Div \left(U H\right) = a$$
    $$\frac{\partial H}{\partial t} = \Div \left(D \grad s \right) + a$$
where FIXME; Recall $\displaystyle D = \frac{2 A (\rho g)^n}{n+2} H^{n+2} |\grad s|^{n-1}$ where $A$ is ice softness, $\rho$ is ice density, $g$ is gravity, $H$ is ice thickness, and $s$ is ice surface elevation; TSUNAMI WAVES EXAMPLE OF FORMER NOT LATER; NOTHING LIKE TSUNMAMIS IN ICE BECAUSE ICE FLOWS DOWNHILL WITH RARE EXCEPTIONS

FIXME SITUATION COMPLICATED AND PRONE TO MISUNDERSTANDING; NEED PERFORMANCE MODEL TO UNDERSTAND TRADE-OFFS; We expect that time-stepping glacier and ice-sheet numerical models will become faster, and thus more useful tools for glaciologists, when multigrid is applied to the solution of the problem(s) solved at each time-step.  This hope can be understood and assessed by a \emph{performance model} for the scheme, one which combines computational work estimates for subproblems into an asymptotic, in the limit of highly-refined meshes, description of of the work required for a given simulation task.

\subsection{Performance models}

Consider numerical simulations based on $n$ mesh nodes in the map-plane, i.e.~$n$ total horizontal grid points.  (Table \ref{tab:notation} lists all of our notation.)  Such a simulation uses $O(n)$ surface elevation variables, the \emph{degrees of freedom} in our analysis.  It requires $O(n)$ memory to store the instantaneous model state, including geometry, velocity, and thermodynamical variables, if we assume that the vertical grid/mesh has an \emph{a priori} bounded vertical resolution, a common \citep[for example]{Brinkerhoffetal2017,Hoffmanetal2018}, though not universal \citep{IsaacStadlerGhattas2015}, property of such models.

\begin{table*}[ht]
\begin{tabular}{cll}
\emph{name} & \emph{meaning} & \emph{units} \\ \hline
$\alpha$    & one fixed-geometry Stokes velocity solution requires $O(n^{1+\alpha})${\large \strut} work\\
$\beta$     & one implicit SIA geometry-update requires $O(n^{1+\beta})$ work \\
$\gamma$    & one implicit Stokes geometry-update plus velocity solution requires $O(n^{1+\gamma})$ work \\
$D$         & representative geometric (shallow ice approximation) diffusivity of an ice sheet & $\text{km}^2 \text{a}^{-1}$ \\
$L$         & width of simulation domain & km \\
$n$         & degrees of freedom: number of nodes in the horizontal (flowline or map-plane) mesh \\
$q$         & time steps per model year needed to resolve modeled climate interactions & $\text{a}^{-1}$ \\
$T$         & simulation duration in model years & a \\
$\Delta t$  & length of time step in model years & a \\
$U$         & representative horizontal ice velocity & $\text{km}\,\text{a}^{-1}$ \\
$\Delta x$  & representative width (diameter) of map-plane mesh cells & km
\end{tabular}
\caption{Notation for performance modeling.  Parameters $\alpha,\beta,\gamma,n$ are pure numbers.}
\label{tab:notation}
\end{table*}

The glacier simulation covers a model domain of horizontal span $L>0$.  For concreteness, in 1D the domain is an interval $[0,L]$, while in 2D it is $[0,L]^2$, or a subset thereof.  Let $\Delta x$ be the characteristic width (diameter) of map-plane mesh cells, thus $\Delta x = O(L n^{-1})$ in 1D and $\Delta x = O(L n^{-1/2})$ in 2D.

Ice sheet models involve climate interaction, especially SMB, and the interaction is dominated by the annual cycle.  Let $q$ be the number of time steps per year needed to incorporate this climate interaction.  Typical values $q=1,12,365 \,\text{a}^{-1}$ correspond to yearly, monthly, and daily climatic interaction frequency.

Almost all current-technology glacier and ice-sheet models use explicit time-stepping which is only conditionally stable \citep{LeVeque2007}.  We will assume that for the resolutions used in scientific applications, maintenance of the stability of explicit time-stepping requires substantially more than $Q$ time steps per year.

For an explicit SIA model the well-known stability restriction is $\Delta t < O(D^{-1} \Delta x^2)$ where $D$ is a representative effective diffusivity in $\text{km}^2\,\text{a}^{-1}$ \citep{Bueleretal2005,HindmarshPayne1996}.
For Stokes dynamics the question of explicit time-stepping stability is largely unexplored.  We propose a \emph{pessimistic} explicit time-step stability paradigm, namely the above SIA restriction $\Delta t < O(D^{-1} \Delta x^2)$.  This is motivated by the small geometric aspect ratio analysis of the Stokes model leading to the SIA model \citep{GreveBlatter2009}.  Alternatively one might subscribe to an \emph{optimistic} paradigm supposing that stability is controlled by the advective restriction $\Delta t < O(U^{-1} \Delta x)$ for some typical horizontal velocity scale $U$ in $\text{km}\,\text{a}^{-1}$.

So-called ``higher-order'' and ``hybrid'' are perhaps better-studied than numerical Stokes models, especially over a large range of horizontal resolutions.  Certain ``hybrid'' schemes simply use the SIA-type stability restriction \citep{Winkelmannetal2011}.  The optimistic concept above is partially supported by a recent analysis of a certain higher-order scheme (``DIVA'') for which $\Delta t < \min\left\{C,O(U^{-1} \Delta x)\right\}$ \citep[equations (52) and (56)]{Robinsonetal2022}, wherein $C>0$ is independent of $\Delta x$.  However, practical Greenland DIVA simulations in the same work actually suggest $\Delta t = O(\Delta x^{1.6})$ instead \citep[Figure 3(a)]{Robinsonetal2022}.

For fixed geometry, as in each time step of an explicit simulation, the momentum balance equations generally require nontrivial solutions as PDEs.  For the Stokes model we suppose one such solution requires $O(n^{1+\alpha})$ work.  Note that such a scheme will determine the surface velocity appearing in the SKE.  However, the value of $\alpha \ge 0$ depends strongly on the numerical velocity/pressure solver implementation, that is, the numerics by which the Stokes momentum balance is actually solved.  Naive solution might involve $\alpha=2$, for example if dense linear algebra is involved, but application of geometric multigrid methods by \cite{IsaacStadlerGhattas2015}, with a solver specifically adapted to the geometry and flow law applicable to ice sheets, suggest that $\alpha \approx 0$ is potentially attainable.  (Of course, the constant $C$ in $(\text{work}) \le C n^{1+\alpha}$ is large.)  Geometric multigrid methods have also been applied to the higher-order (hydrostatic) momentum balance equations \citep{BrownSmithAhmadia2013}, a shallow approximation which resolves both vertical shear stresses and membrane stresses, and in this case a velocity solution work estimate $O(n^{1+\alpha})$ with $\alpha \approx 0$ is a reasonable aspiration.

The SIA model velocity computation is a trivialization of the Stokes problem.  Each velocity unknown is computed in $O(1)$ work, thus the velocity solution is attained (from fixed geometry) in $O(n)$ work.

FIXME \citep{LofgrenAhlkronaHelanow2021} suggest how to reduce the constant in explicit methods for Stokes geometry updates

On the other hand, an unconditionally-stable implicit geometric-update scheme, e.g.~as demonstrated in \citep{Bueler2016} for the SIA, will take at most $Q$ steps per model year in order to follow the climate forcing.  For such an implicit scheme, our stability model is that there is no restriction in the fine-resolution limit ($\Delta x \to 0$).  For example, the steady-state SIA solver in \cite{Bueler2016} uses $\Delta t=+\infty$, with $\Delta t = 100 \,\text{a}$ as a recovery strategy in the case of solver non-convergence.

At each time step of an ice sheet simulation the geometry is updated, for example the old surface elevation values are replaced by new ones.  For an explicit stepping scheme the work in this update is small, specifically $O(n)$, because, once the ice velocity is computed from fixed geometry by the chosen momentum balance model, each grid point requires $O(1)$ work to update.

Table \ref{tab:performancemodel} summarizes our performance models.

\newcommand{\oo}[1]{{\LARGE \strut} \displaystyle O\left(#1\right)}
\setlength{\tabcolsep}{5pt}
\renewcommand{\arraystretch}{1.5}
\begin{table*}[ht]
\begin{tabular}{lll}
\emph{time-stepping} & \emph{dynamics} & \emph{work (flops per model year)} \\ \hline
explicit & SIA    & 1D:\quad $\oo{\frac{D}{L^2} n^3}$ \\
         &        & 2D:\quad $\oo{\frac{D}{L^2} n^2}$ \\
explicit & Stokes & 1D:\quad $\oo{\frac{U}{L} n^{2+\alpha}}$ \quad $\left[\oo{\frac{D}{L^2} n^{3+\alpha}} \text{if pessimistic}\right]$ \\
         &        & 2D:\quad $\oo{\frac{U}{L} n^{1.5+\alpha}}$ \quad $\left[\oo{\frac{D}{L^2} n^{2+\alpha}} \text{if pessimistic}\right]$ \\
implicit & SIA    & $\oo{q\, n^{1+\beta}}$ \\
implicit & Stokes & $\oo{q\, n^{1+\gamma}}$
\end{tabular}
\caption{Asymptotic scaling of computational work for time-stepping numerical ice sheet simulations as the degrees of freedom $n$ goes to infinity (high horizontal resolution).  Note 1D and 2D refer to flow-line and map-plane simulations, respectively.  See Table \ref{tab:notation} for remaining notation.}
\label{tab:performancemodel}
\end{table*}

We assume that the simulation is over a time interval $[0,T]$ where $T>0$ is the \emph{duration} in model years.  For explicit methods note that $\Delta t = T/m$ for $m$ actual, stability-limited time steps in the simulation.  That is, we may define $m=T/\Delta t$ as the number of actual time steps taken.

Combining our observations about the scaling of $\Delta x$ with $n$, for explicit schemes we have
\begin{equation}
\Delta t^{-1} = \frac{m}{T} > O\left(\frac{D n^2}{L^2}\right)
\end{equation}
in the 1D SIA and pessimistic-Stokes cases and
\begin{equation}
\Delta t^{-1} > O\left(\frac{D n}{L^2}\right)
\end{equation}
in the corresponding 2D cases.  (Recall that the power on $n$ is lower in 2D cases because $\Delta x = L/\sqrt{n}$.)  In the optimistic-Stokes case these estimates become
\begin{equation}
\Delta t^{-1} > O\left(\frac{U n}{L}\right), O\left(\frac{U \sqrt{n}}{L}\right)
\end{equation}
in 1D and 2D respectively.

This estimate of $\Delta t^{-1}$ is multiplied by the per-step expense to give a work estimate for each model year in a simulation.  The results are shown in the ``explicit'' rows of Table \ref{tab:performancemodel}.  Note that 2D simulationsare much more expensive for a given resolution $\Delta x$ because $n$ is so much larger; $n = O(\Delta x^{-1})$ in 1D while $n = O(\Delta x^{-2})$ in 2D.

For unconditionally-stable implicit methods, however, $\Delta t$ does not need scale with grid resolution; it has fixed value $\Delta t = 1/q$ model years.  That is, the time steps are determined only by the need to resolve climatic interactions.

We suppose that the work per geometry-update solution of an implicit SIA simulation is $O(n^{1+\beta})$ for some $\beta>0$.  The \cite{Bueler2016} geometry-update method is fundamentally not a multigrid method, and it has FIXME WHAT $\beta$?

For an implicit Stokes simulation the geometry/velocity solve is assumed to be, in the absense of constraining research, $O(n^{1+\gamma})$ for some $\gamma>0$ to be determined.  The nMCD method proposed in [Bueler 2022] is expensive but multigrid, so it may turn out that $\gamma$ is small.

Noting that $n$ is the number of horizontal grid points, the work estimate for an implicit time step may not depend on the dimension, but in fact it is likely that $\alpha,\beta,\gamma$ each depend on dimension in some manner; this should be informed by further research.  Furthermore, the estimate reported in Table \ref{tab:performancemodel} must be multiplied by a potentially very large scheme-dependent constant.

\subsection{Discussion and Conclusion}

Existing multigrid methods for the velocity solution process show $\alpha$ is close to zero.  That is, the evidence shows that $\alpha$ is small for well-engineered Stokes \citep{IsaacStadlerGhattas2015} and higher-order \citep{BrownSmithAhmadia2013} solvers.

The promise of multigrid methods for the implicit, coupled geometry-update plus velocity solution problems is to also make $\beta,\gamma$ smaller.  However, this aspiration is long-term because work is barely started on it.

%         References
\bibliography{perfmod}
\bibliographystyle{igs}

\end{document}
